%!TEX program = xelatex
%!TEX encoding = UTF-8 412-268-2097Unicode

\documentclass[12pt]{article}
\usepackage{geometry}                % See geometry.pdf to learn the layout options. There are lots.
\geometry{letterpaper}                   % ... or a4paper or a5paper or ... 
%\geometry{landscape}                % Activate for for rotated page geometry
%\usepackage[parfill]{parskip}    % Activate to begin paragraphs with an empty line rather than an indent
\usepackage{graphicx}
\usepackage{amssymb}
\usepackage{amsmath}

% Will Robertson's fontspec.sty can be used to simplify font choices.
% To experiment, open /Applications/Font Book to examine the fonts provided on Mac OS X,
% and change "Hoefler Text" to any of these choices.

\usepackage{fontspec,xltxtra,xunicode}

\newcommand{\vect}[1]{\boldsymbol{#1}}

\defaultfontfeatures{Mapping=tex-text}
\setromanfont[Mapping=tex-text]{Hoefler Text}
\setsansfont[Scale=MatchLowercase,Mapping=tex-text]{Gill Sans}
\setmonofont[Scale=MatchLowercase]{Andale Mono}

\title{Solutions for Chapter 7 of \emph{Natural Image Statistics}}
\author{Yimeng Zhang}
%\date{}                                           % Activate to display a given date or no date

\begin{document}
\maketitle

% For many users, the previous commands will be enough.
% If you want to directly input Unicode, add an Input Menu or Keyboard to the menu bar 
% using the International Panel in System Preferences.
% Unicode must be typeset using a font containing the appropriate characters.
% Remove the comment signs below for examples.

% \newfontfamily{\A}{Geeza Pro}
% \newfontfamily{\H}[Scale=0.9]{Lucida Grande}
% \newfontfamily{\J}[Scale=0.85]{Osaka}

% Here are some multilingual Unicode fonts: this is Arabic text: {\A السلام عليكم}, this is Hebrew: {\H שלום}, 
% and here's some Japanese: {\J 今日は}.


\section{Mathematical Exercises} % (fold)
\label{sec:mathematical_exercises}

\subsection{7.14.1} % (fold)
\label{sub:7_14_1}
Trivial. Because $p(x,y) = p(x)p(y)$, and we can separate the integration on these two variables.
% subsection 7_14_1 (end)

\subsection{7.14.2} % (fold)
\label{sub:7_14_2}
Trivial. That's the numerator of correlation coefficient.
% subsection 7_14_2 (end)

\subsection{7.14.3-7.14-6} % (fold)
\label{sub:7_14_3_7_14_6}
Too mathematical...
% subsection 7_14_3_7_14_6 (end)

\subsection{7.14.7} % (fold)
\label{sub:7_14_7}
Trivial... $C = A I A^T = A A^T$.
% subsection 7_14_7 (end)

\end{document}  